\section*{Modelling the Garland}

The garland can be modelled as a circular conical spiral, and applying lower and upper bounds, it has the parametric function \autocite{rejbrandConicalHelix}:
\begin{equation}
    C(t) = \begin{pmatrix}[0.7]
        at\cos{t} \\
        at\sin{t} \\
        bt
    \end{pmatrix}, \quad \forall t \in \left[0, \frac{H}{b}\right]
\end{equation}
where the constants $a, b \in \Real$. Lower and upper bounds are applied because the cone has height $H$, and we only want parts of the curve where $z(t)$ is between $0$ and $H$. Thus, $0 \leq bt \leq H$, and dividing by $b$, we find that $0 \leq t \leq \frac{H}{b}$, which are the bounds applied to $C(t)$. $t$ is the independent variable, and for any given value of $t$, the parametric function outputs a point on the curve, and as we generate an infinite
amount of points from the lower bound to the upper bound, the locus of points formed generate the shape of the space curve. The z-axis of the spiral coincides with the tree's axis of radial symmetry and $C(0)$ represents the tip of the tree.

To ensure that the spiral sits on the surface of the cone (the Christmas tree), it is necessary to find appropriate values for $a$ and $b$. Using methodology inspired by a blog post authored by Stewart and Heighway, we first define the radial distance $\rho(t)$ as the distance of a point on the curve to the tree's axis of radial symmetry, visualized in Figure \ref{fig:radial}. By Pythagorean's theorem:
\begin{equation*}
    \rho(t) = \sqrt{x(t)^2+y(t)^2}
\end{equation*}
\begin{wrapfigure}{r}{0.4\textwidth}
    \centering
    \includegraphics[width=0.38\textwidth]{diagrams/radial_distance.pdf}
    \caption{Radial distance of a point on the spiral to the z-axis (top-down view).} \label{fig:radial}
    \vspace*{-30pt}
\end{wrapfigure}
\bulletarrow Substituting for $x(t)$ and $y(t)$:
\begin{align}
    \rho(t) & = \sqrt{a^2t^2\cos^2{t}+a^2t^2\sin^2{t}} \notag \\
            & = at\sqrt{\cos^2{t}+\sin^2{t}} \notag           \\
            & = at
\end{align}

The radial distance is useful because the ratio of $\rho(t)$ over $z(t)$ for any given $t$ will always be proportional to $H$ over $R$ for any given $t$. This because any valid point of the curve should sit on the surface of the cone of height $H$ and radius $R$, and thus the radial distance $\rho(t)$ and vertical distance $z(t)$ of the point would form a right triangle as visualized in Figure \ref{fig:sim_tri}, which would be similar to the triangle formed by the vertical cross-section of the cone by angle-angle, due to the shared an interior angle. This allows us to establish the following proportional relationship:
\begin{gather}
    \frac{R}{H} = \frac{\rho(t)}{z(t)} = \frac{at}{bt} \notag \\
    \Rightarrow \frac{R}{H} = \frac{a}{b} \label{eq:proportion}
\end{gather}

\begin{wrapfigure}[7]{l}{0.3\textwidth}
    \includegraphics[width=0.28\textwidth]{diagrams/similar_triangles.pdf}
    \caption{Similar Triangles.} \label{fig:sim_tri}
\end{wrapfigure}
In other words, for the spiral to lie on the surface of the cone with radius $R$ and height $H$, the ratio $a$ over $b$ must be proportional to $R$ over $H$. This relationship is visualized in Figure \ref{fig:param_comparison}, where we can see that larger values for $a$ and $b$ correspond to larger spacing between successive rotations of garland, while smaller values lead to smaller spacing between successive rotations of the garland. From this, we can establish a relationship between $\lambda$, $a$, and $b$.

\begin{figure}[H]
    \centering
    \begin{subfigure}[t]{0.32\textwidth}
        \centering
        \includegraphics[width=\textwidth]{images/a_vs_b/close.png}
        \caption{$a=0.075$, $b=0.360$}
    \end{subfigure}
    \begin{subfigure}[t]{0.32\textwidth}
        \centering
        \includegraphics[width=\textwidth]{images/a_vs_b/medium.png}
        \caption{$a=0.225$, $b=1.08$}
    \end{subfigure}
    \begin{subfigure}[t]{0.32\textwidth}
        \centering
        \includegraphics[width=\textwidth]{images/a_vs_b/sparse.png}
        \caption{$a=0.450$, $b=2.16$}
    \end{subfigure}
    \caption{Spirals that have the same ratio of $\frac{a}{b}$ lie on the same cone} \label{fig:param_comparison}
\end{figure}

For every rotation of the garland, $t$ increases by $2\pi$ because that is the period of the trigonometric functions sine and cosine. Thus, the change in radial distance, $\Delta \rho$, and change in vertical distance, $\Delta z$, after one full period would be:
\begin{equation*}
    \begin{aligned}[c]
        \Delta \rho & = \rho(t+2\pi) -\rho(t) \\
                    & = a\cdot(t+2\pi) - at   \\
                    & = 2\pi a
    \end{aligned}
    \qquad\qquad\qquad\qquad
    \begin{aligned}[c]
        \Delta z & = z(t+2\pi) -z(t)     \\
                 & = b\cdot(t+2\pi) - bt \\
                 & = 2\pi b
    \end{aligned}
\end{equation*}
While $\lambda$ represents the distance between consecutive rotations of garland, we can also think of it as the change in position of the spiral along of the slant length of the cone per rotation. Thus, by the Pythagorean theorem:
\begin{equation}
    \lambda^2 = 4\pi^2a^2+4\pi^2b^2 \label{eq:spacing}
\end{equation}
With this, we now have 2 equations with $a$ and $b$, and we can represent $a$ and $b$ in terms of our chosen parameters. From equation \ref{eq:proportion}, we can isolate $b$ to get that $b = \frac{H}{R}a$, which can be substituted back into equation \ref{eq:spacing}:
\begin{equation*}
    \Rightarrow \lambda^2 = 4\pi^2a^2+\frac{4\pi^2a^2H^2}{R^2}  = 4\pi^2a^2\left(1+\frac{H^2}{R^2} \right)
\end{equation*}
\bulletarrow Isolating for $a$:
\begin{equation*}
    a = \sqrt{\frac{\lambda^2}{4\pi^2(1+\frac{H^2}{R^2})}} = \frac{\lambda}{2\pi\sqrt{1+\frac{H^2}{R^2}}} = \frac{\lambda R}{2\pi\sqrt{R^2+H^2}}
\end{equation*}
\bulletarrow Recognizing that $S=\sqrt{R^2+H^2}$ is the slant height of the cone:
\begin{equation}
    a =\frac{\lambda R}{2\pi S}
\end{equation}
\bulletarrow Plugging this back in equation \ref{eq:proportion}, we get:\begin{equation}
    b =\frac{\lambda H}{2\pi S}
\end{equation}
Thus, we finally have that the parametric equation for the garland is:
\begin{equation}
    C(t) = \frac{\lambda}{2\pi S}
    \begin{pmatrix}[0.7]
        Rt\cos{t} \\
        Rt\sin{t} \\
        Ht
    \end{pmatrix}, \quad \forall t \in \left[0, \frac{2\pi S}{\lambda}\right]
\end{equation}
