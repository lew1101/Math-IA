\section{Conclusion and Evaluation}

In this investigation, I was successful in deriving a formula which calculates the length of garland required to wrap around my Christmas tree based on the spacing between successive rotations of garland $\lambda$, the base radius of the tree $R$, and the height of the tree $H$. This was done by modelling the garland as a circular conical spiral, and by finding the arc length of the spiral using integration, I was able to find the approximate length garland I would need based on my chosen parameters. Then, I applied this formula by plugging in the dimensions of my tree, and found optimal values of $\lambda$ by only considering solutions precise up to a \sfrac{1}{4} of an inch, and subjected the function to further constraints -- the amount of wasted garland must be less than 15\% of the length of a strand of garland ($\US{12}{\inch}$), and potential values of $\lambda$ must be between $\qtyrange{8.00}{10.00}{\inch}$. I selected these two constraints rather arbitrarily to satisfy each of my respective aims: minimizing waste and aligning with my personal aesthetic preferences. There were two solutions which met these constraints, and I ultimately chose $\lambda=\US{10}{\inch}$, requiring the use of 5 strands of garland that are $\US{72}{\inch}$ long, totaling in $\US{360}{\inch}$ or $\US{30}{\feet}$ of garland. 

While I was able to meet all of my objectives in my investigation and find an ``optimal'' answer, it is also important to realize that my investigation was hinged upon the two big assumptions that I made in the beginning of my investigation -- that the tree can be modelled as a cone, and that the garland wraps the cone perfectly as a circular conical spiral. In reality, trees will always have imperfections, and the garland will sag as it travels up the tree. It is also difficult to account for how each person wraps the garland around the tree differently, as some people may wrap the garland more tightly around the tree, while others may wrap it around looser. These differences can add up, leading to large variations in the amount of garland used, and considering that I chose a solution which had effectively no remaining excess garland ($\US{0.01}{\inch}$), there is very little room for error. The formula is also rather complex, meaning that it is not really accessible to those who are not mathematically inclined. As such, I can make my formula more accessible by approximating it using simpler functions, as well as potentially accounting for some leeway (such as by including a scale factor), as it would be better to have an overestimation rather than an underestimation. An assessment of how this might be achieved can be a possible extension to this study.
