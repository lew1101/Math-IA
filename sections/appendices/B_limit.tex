\section{Solving the Limit of $L(\lambda, R, H)$ as $\lambda \to +\infty$} \label{sec:qzhsb}

\begin{align*}
     \lim_{\lambda \to +\infty} &\left(\frac{\sqrt{R^2+H^2}}{2\lambda}\sqrt{\lambda^2+4\pi^2R^2}+\frac{\lambda \sqrt{R^2+H^2}}{4\pi R}\bigg[\ln\left(\sqrt{\lambda^2+4\pi^2R^2}+2\pi R\right)-\ln\lambda\bigg]\right) \\ 
    &= \lim_{\lambda \to +\infty} \frac{\sqrt{R^2+H^2}}{2\lambda}\sqrt{\lambda^2+4\pi^2R^2}+ \lim_{\lambda \to +\infty}\frac{\lambda \sqrt{R^2+H^2}}{4\pi R}\bigg[\ln\left(\sqrt{\lambda^2+4\pi^2R^2}+2\pi R\right)-\ln\lambda\bigg] \\ 
    &= \lim_{\lambda \to +\infty} \frac{\cancel{\lambda}\sqrt{R^2+H^2}}{2\cancel{\lambda}} \cdot \cancelto{1}{\lim_{\lambda \to +\infty} \sqrt{1+\frac{4\pi^2R^2}{\lambda^2}}} + \lim_{\lambda \to +\infty}\frac{\lambda \sqrt{R^2+H^2}}{4\pi R}\bigg[\ln\left(\sqrt{\lambda^2+4\pi^2R^2}+2\pi R\right)-\ln\lambda\bigg] \\
    &=\frac{\sqrt{R^2+H^2}}{2} + \lim_{\lambda \to +\infty} \frac{\ln\left(\sqrt{\lambda^2+4\pi^2R^2}+2\pi R\right)-\ln\lambda}{\frac{4\pi R}{\lambda \sqrt{R^2+H^2}}} \\
\intertext{\bulletarrow{Since the limit of the second term evaluates to $\frac{0}{0}$, by l'H\^{o}pital's rule:}}
    &= \frac{\sqrt{R^2+H^2}}{2} + \lim_{\lambda \to +\infty} \frac{\dv{\lambda}\left(\ln\left(\sqrt{\lambda^2+4\pi^2R^2}+2\pi R\right)-\ln\lambda\right)}{\dv{\lambda}\left(\frac{4\pi R}{\lambda \sqrt{R^2+H^2}}\right)} \\ 
    &= \frac{\sqrt{R^2+H^2}}{2} + \lim_{\lambda \to +\infty} \frac{\frac{-1}{\lambda}+\frac{\lambda}{2\pi R\sqrt{\lambda^2+4\pi^2R^2}+4\pi^2R^2+\lambda^2}}{\frac{-4\pi R}{\lambda^2\sqrt{R^2+H^2}}} \\ 
\intertext{\bulletarrow{Combining the fractions in the limit by taking the common denominator:}}
    &= \frac{\sqrt{R^2+H^2}}{2} + \lim_{\lambda \to +\infty} \left(\frac{-\lambda^2\sqrt{R^2+H^2}}{4\pi R} \cdot \frac{\cancel{\lambda^2}-2\pi R\sqrt{\lambda^2+4\pi^2R^2}-4\pi^2R^2\cancel{-\lambda^2}}{2\pi R\lambda\sqrt{\lambda^2+4\pi^2R^2}+4\pi^2R^2\lambda+\lambda^3}\right) \\
    &= \frac{\sqrt{R^2+H^2}}{2} + \lim_{\lambda \to +\infty} \left(\frac{\lambda^2\sqrt{R^2+H^2}}{4\pi R} \cdot \frac{2\pi R\lambda\sqrt{1+\frac{4\pi^2R^2}{\lambda^2}}+4\pi^2R^2}{2\pi R\lambda^2\sqrt{1+\frac{4\pi^2R^2}{\lambda^2}}+4\pi^2R^2\lambda+\lambda^3}\right) \\ 
\intertext{\bulletarrow{Multiplying the numerator and denominator by $\frac{1}{\lambda^3}$:}}
    &= \frac{\sqrt{R^2+H^2}}{2} + \lim_{\lambda \to +\infty} \left(\frac{\sqrt{R^2+H^2}}{4\pi R} \cdot \frac{2\pi R\sqrt{1+\frac{4\pi^2R^2}{\lambda^2}}+\frac{4\pi^2R^2}{\lambda}}{\frac{2\pi R\sqrt{1+\frac{4\pi^2R^2}{\lambda^2}}}{\lambda}+\frac{4\pi^2R^2}{\lambda^2}+1}\right) \\ 
    &= \frac{\sqrt{R^2+H^2}}{2} + \frac{\sqrt{R^2+H^2}}{2} \cdot \cancelto{1}{\lim_{\lambda \to +\infty} \frac{\sqrt{1+\frac{4\pi^2R^2}{\lambda^2}}+\frac{2\pi R}{\lambda}}{\frac{2\pi R\sqrt{1+\frac{4\pi^2R^2}{\lambda^2}}}{\lambda}+\frac{4\pi^2R^2}{\lambda^2}+1}} \\ 
    &\boxed{=\sqrt{R^2+H^2}}
\end{align*}
Thus, the limit of $L(\lambda, R. H)$ as $\lambda \to +\infty$ will always be equal to the slant height of the tree.
